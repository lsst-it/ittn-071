\documentclass[PMO,authoryear,toc]{lsstdoc}
\input{meta}

% Package imports go here.

% Local commands go here.

%If you want glossaries
%\input{aglossary.tex}
%\makeglossaries

\title{rke Update Playbook}

% Optional subtitle
% \setDocSubtitle{A subtitle}

\author{%
Carlos Barr\'ia
}

\setDocRef{ITTN-071}
\setDocUpstreamLocation{\url{https://github.com/lsst-it/ittn-071}}

\date{\today}

% Optional: name of the document's curator
% \setDocCurator{The Curator of this Document}

\setDocAbstract{%
following document details the procedure to update the current version of `rke` into a newer version with the objective of moving forward with the Kubernetes version installed in the designated cluster.
}

% Change history defined here.
% Order: oldest first.
% Fields: VERSION, DATE, DESCRIPTION, OWNER NAME.
% See LPM-51 for version number policy.
\setDocChangeRecord{%
  \addtohist{1}{YYYY-MM-DD}{Unreleased.}{Carlos Barría}
}

\usepackage{listings}
\lstset{language=Bash,basicstyle=\ttfamily\small}

\begin{document}

% Create the title page.
\maketitle
% Frequently for a technote we do not want a title page  uncomment this to remove the title page and changelog.
% use \mkshorttitle to remove the extra pages

% ADD CONTENT HERE
% You can also use the \input command to include several content files.
\section{Introduction}


In the following document we will explain step by step the current procedure to upgrade the version of Kubernetes through rke (Rancher Kubernetes Engine), which is the current tool we use to assemble our Kubernetes Clusters.

It is very important to try to keep up with the current version of this software as it can:
\begin{itemize}
\item
  Apply security patches to the software.
\item
  Apply bug fixes to solve issues from the past versions
\item
  Improve performance in some cases.
\item
  Add new features that could improve utilization of the product.
\end{itemize}
\newpage

\section{Playbook}

\begin{enumerate}
  \item First, we need to open a Jira Ticket to attach our work and set the state to In Progress.
  \item Clone or update your current version of both repositories in lsst github account lsst-control and lsst-puppet-hiera-private.
  \item Create a branch on both repositories with the IT ticket number you created. \textit{(ex: IT-3956\_rke\_updates)}
  \item Update the rke version to match the desired Kubernetes version.
  \begin{enumerate}[label=(\alph*)]
    \item Check the rke github repository for the release information \href{https://github.com/rancher/rke/releases}{rke releases}
      \textit{Note: rke and k8s versions are different, you need to check that the version of rke supports the version of k8s you need to upgrade to.}
    \item In the rke releases page, search for the kubernetes version you want to upgrade and notice which rke version supports it.
      \textit{ex: Kubernetes version v1.25.9-rancher2-1 needs rke version v1.4.6-rc2}
      \textit{use: } \verb|rke config -l -a| \textit{to see supported versions on the current installed rke}
    \item Update the rke version in the hierarchy on the lsst-control repository and push those changes. 
          (ex: \href{https://github.com/lsst-it/lsst-control/blob/6a2fa54f9e8eccb3dfe178e0234cd9546ddf0b3d/hieradata/site/cp/role/rke.yaml#L2}{rke.yaml})
    \item If this is a new version of rke that's never been used, we need to add the checksum of such binaries to the \textit{profile::core::rke} module.
          (ex: \href{https://github.com/lsst-it/lsst-control/blob/6a2fa54f9e8eccb3dfe178e0234cd9546ddf0b3d/site/profile/manifests/core/rke.pp#L43-L47}{rke.pp})
  \end{enumerate}
  \item Once the changes have been pushed, change all the hosts that belong to the cluster into the environment of your branch.
  \begin{enumerate}[label=(\alph*)]
    \item Search with \verb|params.role = rke| select the cluster hosts.
    \item Force puppet to run in those nodes.
  \end{enumerate}
  \item Connect to the head node of the cluster via ssh.
  \item Switch into the user rke with \verb|sudo -iu rke|
  \item Pre-updates sanity check:
  \begin{enumerate}[label=(\alph*)]
    \item Verify the nodes of the cluster are online and ready. \verb|kubectl get nodes|
    \item Check new version of the rke client. \verb|rke --version|
    \item If the cluster has \textit{rook-ceph} deployed check on the health status.
  \end{enumerate}
  \item Switch to the \textit{k8s-cookbook} repository and update the \textit{cluster.yaml} file in the \textit{rke} folder of the cluster.
  \begin{enumerate}[label=(\alph*)]
    \item Update your local repository or clone the \textit{k8s-cookbook}.\\
      \verb|ex: git clone https://github.com/lsst-it/k8s-cookbook|
    \item Create a branch on this repository with the same name used in the control repository.
      \verb|git checkout -b IT-3956_rke_updates|
    \item Edit the cluster.yaml file inside the rke folder on the selected k8s cluster and change it to the desired version.
      \verb|ex: kubernetes_version: "v1.25.9-rancher2-1"|
    \item Stage the file, commit your changes and push them into the remote branch.\\
      \verb|ex: git add cluster.yaml; git commit -m "(ruka) bump k8s to v1.59.9-rancher2-1"; git push|
    \item Log into the cluster and switch to the rke user.
      \verb|ex: sudo -iu rke|
    \item Update the current repository and switch to your branch.\\
      \verb|ex: git fetch --all --prune; git checkout IT-3956_rke_updates|
    \item Ensure the repository is clean with no local changes and the commit is correct.
      \verb|ex: git status; git log -p|
  \end{enumerate}
  \item Announce the update on the relevant Slack channel before proceeding further.
  \item Set yourself on the rke folder on the selected cluster and run:
  \begin{enumerate}[label=(\alph*)]
    \item Save current config with \verb|make tarball|.
    \item Create a snapshot of the current etcd \verb|make snapshot|.
    \item Update the version. \verb|rku up|
    \item Watch the nodes get updated. \verb|kubectl get nodes -w|
    \item When the process is done is good to run \verb|rke up| again to check the process was idempotent.
  \end{enumerate}
  \item Validate and check status on the following:
  \begin{enumerate}[label=(\alph*)]
    \item Pods rollouts and broken states with \verb|kubectl get pods -A -w|
    \textit{Note: canal changes may take a while to rollout}
    \item Deployments on the kube-system namespace with \verb|kubectl -n kube-system get deployments|
    \item Deamonset status on the kube-system ns with \verb|kubectl -n kube-system get deamonset|
    \item Check the status of the rook-ceph deployment if the cluster has one.
    \textit{Note: Through the dashboard or inside the rook-ceph-tool pod with} \verb|ceph status|
  \end{enumerate}
  \item Grab the status of the cluster and report back into the JIRA Ticket.
  \item Announce on the correct Slack channel that the update is complete.
  \item Open both branches pull request to merge this new changes into the production branch.
  \textit{Note: PRs on k8s-cookbook and lsst-control repositories}
  \item Merged both changes into production and delete the branch created in the
    \verb|lsst-puppet-hiera-private|
  \item If the cluster was in the summit then roll forward the \verb|cp_production| branch.
  \item Go back into foreman and change the corresponding hosts back to the production environment.
  \item Finally, close the Jira Ticket.
\end{enumerate}
\newpage

\appendix
% Include all the relevant bib files.
% https://lsst-texmf.lsst.io/lsstdoc.html#bibliographies
\section{References} \label{sec:bib}
\renewcommand{\refname}{} % Suppress default Bibliography section
\bibliography{local,lsst,lsst-dm,refs_ads,refs,books}

% Make sure lsst-texmf/bin/generateAcronyms.py is in your path
\section{Acronyms} \label{sec:acronyms}
\addtocounter{table}{-1}
\begin{longtable}{p{0.145\textwidth}p{0.8\textwidth}}\hline
\textbf{Acronym} & \textbf{Description}  \\\hline

DM & Data Management \\\hline
\end{longtable}

% If you want glossary uncomment below -- comment out the two lines above
%\printglossaries
\end{document}
